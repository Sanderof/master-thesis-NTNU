\chapter{Introduction}\label{sec:introduction}

\todo{Shortly present context/motivation of master thesis}

\begin{comment}
    Not phrase as goals, but talk about what we have done
    Discuss which open questions there are in this domain instead of finding research questions

    Motivation should be explained in introduction
\end{comment}

% \section{Project report goals}
% \todo{Describe the main goals/purposes of this project report}
% \begin{enumerate}
%     \item Collect data
%     \item Literature review to find what is done and what we can do next semester
% \end{enumerate}



\begin{comment}
The introduction of the thesis should take the reader all the way from the big picture and context of the project to the concrete task that has been solved in the thesis. A nice skeleton for a good introduction was given by \textcite{claerbout1991scrutiny}: \emph{review–claim–agenda}. In the review part, the background of the project is covered. This leads up to your claim, which is typically that some entity (software, device) or knowledge (research questions) is missing and sorely needed. The agenda part briefly summarises how your thesis contributes.
\end{comment}

