\chapter{Introduction}\label{sec:introduction}

This project report is written as part of the course \textit{TDT4501 - Computer Science, Specialization Project} at the \textit{Norwegian University of Science and Technology} (NTNU) during the autumn semester 2024. The course serves as prepartion for the master's thesis and allows students to gain experience with researching a topic using scientific working methods. We are two Computer science master's students who have carried out the project and authored this report. The project is carried out as preparation for the master's thesis ``AI in the classroom''. Our main supervisor is \textit{Özlem Özgöbek} and our co-supervisor is \textit{George Adrian Stoica}, both associate professors at the department of computer science. We are also collaborating with PhD candidate \textit{Talha Mahboob Alam}. On the side of the project, we have also followed the two theory modules \textit{TDT06 - Educational Technology} and \textit{TDT07 - Learning Analytics} as part of \textit{TDT4506 - Computer Science, Specialization Course}.

The ``AI in the classroom'' master's thesis can be seen as follow-up work from \cite{stoica2016} and \cite{stoica2017}, co-authored by our co-supervisor in 2016 and 2017, respectively. We will come back to these papers in chapter \ref{chap:relatedwork}, but their purpose was to develop a system for visualising open-text responses from students in real-time to provide the instructor with insights. Such a system may be referred to as a student response system - among several other synonyms. These systems have a long history, but they have mostly been used with closed-ended questions to gather responses from students, as these are easier to quickly analyse. Closed-ended questions do, however, have several limitations, and both the instructor and the students could benefit from open-ended questions instead that require students to write responses in the form of short text - which we will refer to as ``open-text responses''. While open-text responses can be quickly read through by an instructor in small classrooms, this gets close to impossible in large classrooms or lecture halls. With the development of ever more performant AI and text mining methods, it seems quite likely that such methods can be applied, in combination with a student response system and clever visualisations, to help instructors quickly make sense of large amounts of open-text responses in a live class or lecture setting. The investigation into which and how modern text mining methods can be applied to open-text responses to provide instructors with actionable feedback in real-time, as well as the implementation and evaluation of a suitable system, is what defines the master's thesis. In other words, bringing AI into the classroom to improve the learning process.

For the structure of this report, we start by presenting some background for the project and master's thesis in chapter \ref{chap:background}. We will then go through some related work in chapter \ref{chap:relatedwork}. Chapter \ref{chap:method} explains the methods and procedures we have followed for collecting data and performing a systematic  literature review. Chapter \ref{chap:discussion} contains the results of our work and investigation related to data collection, the sytematic literature review, related proprietary solutions, open problems in the domain and planning of the master's thesis itself. To round off, chapter \ref{chap:conclusion} summarises what we have done for this project and the plan ahead for the upcoming semester.







\begin{comment}
The introduction of the thesis should take the reader all the way from the big picture and context of the project to the concrete task that has been solved in the thesis. A nice skeleton for a good introduction was given by \textcite{claerbout1991scrutiny}: \emph{review–claim–agenda}. In the review part, the background of the project is covered. This leads up to your claim, which is typically that some entity (software, device) or knowledge (research questions) is missing and sorely needed. The agenda part briefly summarises how your thesis contributes.
\end{comment}

