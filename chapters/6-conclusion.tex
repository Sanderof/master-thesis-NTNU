\chapter{Conclusion}\label{chap:conclusion}

% Keep very short about we have done in this report
% Paragraph about what we will do next semester

In this project report, written as part of \textit{TDT4501 - Computer Science, Specialization Project}, we have made preparations for the master's thesis ``AI in the classroom'' to be conducted the coming spring semester 2025. we have conducted a data collection process, which has yielded around 1800 open-text responses from students at NTNU. In addition, we have started the work on a systematic literature review together with a PhD candidate. This review will help us shed light on how and which text mining methods are used in education. As an extension to the literature found, we have identified some challenges for the implementation of our own system. Lastly, we have created the outline for a prototype of the system architecture and the evaluation process for the system.

For the master's thesis itself, we will finish the data extraction of the systematic literature and then use the findings to inform experimentation on the collected data with promising text mining methods. While finding and implementing suitable text mining methods will make up the core of our master's thesis, we will also set up a minimum viable product of our proposed prototype. When we eventually have a working system, we will evaluate it by testing it in a live lecture here at NTNU, which will include a recruitment of lecturers and interviewees among the students.

\begin{comment}
The conclusion chapter is usually quite short – a paragraph or two – mainly summarising what was achieved in the project. It should answer the \emph{claim} part of the introduction. It should also say something about what comes next (`future work').
\end{comment}
