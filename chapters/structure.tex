\chapter{Thesis Structure}
The structure of the thesis, i.e., which chapters and other document elements that should be included, depends on several factors such as the study level (bachelor, master, PhD), the type of project it describes (development, research, investigation, consulting), and the diversity (narrow, broad). Thus, there are no exact rules for how to do it, so whatever follows should be taken as guidelines only.

A thesis, like any book or report, can typically be divided into three parts: front matter, body matter, and back matter. Of these, the body matter is by far the most important one, and also the one that varies the most between thesis types.

\section{Front Matter}
\label{sec:frontmatter}

The front matter is everything that comes before the main part of the thesis. It is common to use roman page numbers for this part to indicate this. The minimum required front matter consists of a title page, abstract(s), and a table of contents. A more complete front matter, in a typical order, is as follows.

\begin{description}
    \item[Title page:] The title page should, at minimum, include the thesis title, authors and a date. A more complete title page would also include the name of the study programme, and possibly the thesis supervisor(s). See \cref{sec:setup}.
    \item[Abstracts:] The abstract should be an extremely condensed version of the thesis. Think one sentence with the main message from each of the chapters of the body matter as a starting point. \textcite{landes1951scrutiny} have given some very nice instructions on how to write a good abstract. A thesis from a Norwegian Univeristy should contain abstracts in both Norwegian and English irrespectively of the thesis language (typically with the thesis language coming first).
    \item[Dedication:] If you wish to dedicate the thesis to someone (increasingly common with increasing study level), you may add a separate page with a dedication here. Since a dedication is a personal statement, no template is given. Design it according to your preference.
    \item[Acknowledgements:] If there is someone who deserves a `thank you', you may add acknowledgements here. If so, make it an unnumbered chapter, i.e., \texttt{\textbackslash chapter*\{Acknowledgements\}}.
    \item[Table of contents:] A table of contents should always be present in a document at the size of a thesis. It is generated automatically using the \texttt{\textbackslash tableofcontents} command. The one generated by this document class also contains the front matter and unnumbered chapters.
    \item[List of figures:] If the thesis contains many figures that the reader might want to refer back to, a list of figures can be included here. It is generated using \texttt{\textbackslash listoffigures}.
    \item[List of tables:] If the thesis contains many tables that the reader might want to refer back to, a list of tables can be included here. It is generated using \texttt{\textbackslash listoftables}.
    \item[List of code listings:] If the thesis contains many code listings that the reader might want to refer back to, a list of code listings can be included here. It is generated using \texttt{\textbackslash lstlistoflistings}.
    \item[Other lists:] If there are other list you would like to include, this would be a good place. Examples could be lists of definitions, theorems, nomenclature, abbreviations, glossary etc. There are no standards for this, but many lists can be generated using the \texttt{description} environment (like, e.g., this list of possible front matter content) within a separate \texttt{\textbackslash chapter*\{\}}.
    \item[Preface or Foreword:] A preface or foreword is a good place to make other personal statements that do not fit whithin the body matter. This could be information about the circumstances of the thesis, your motivation for choosing it, or possibly information about an employer or an external company for which it has been written. Again, use, e.g., \texttt{\textbackslash chapter*\{Preface\}}.
\end{description}

\section{Body Matter}

The body matter consists of the main chapters of the thesis. It starts the Arabic page numbering with page~1. There is a great diversity in the structure chosen for different thesis types. Common to almost all is that the first chapter is an introduction, and that the last one is a conclusion followed by the bibliography.

\subsection{Research Project}
\label{sec:resesarch}

For many master and some bachelor projects in computer science, the main task is to gain knew knowledge about something. A thesis describing such a project is typically structed as an extended form of a scientific paper, following the so-called IMRaD (Introduction, Method, Results, and Discussion) model:

\begin{description}
    \item[Introduction:] See \cref{sec:development}.
    \item[Method:] The method chapter should describe in detail which activities you undertake to answer the research questions presented in the introduction, and why they were chosen. This includes detailed descriptions of experiments, surveys, computations, data analysis, statistical tests etc.
    \item[Results:] The results chapter should simply present the results of applying the methods presented in the method chapter without further ado. This chapter will typically contain many graphs, tables, etc. Sometimes it is natural to discuss the results as they are presented, combining them into a `Results and Discussion' chapter, but more often they are kept separate.
    \item[Discussion:] See \cref{sec:development}.
    \item[Conclusion:] See \cref{sec:development}.
    \item[Bibliography:] See \cref{sec:development}.
\end{description}

\section{Back Matter}

Material that does not fit elsewhere, but that you would still like to share with the readers, can be put in appendices. See \cref{app:additional}.
